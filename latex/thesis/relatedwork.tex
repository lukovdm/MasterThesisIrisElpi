\documentclass[thesis.tex]{subfiles}

\ifSubfilesClassLoaded{
  \externaldocument{thesis}
  \setcounter{chapter}{6}
}{}

\begin{document}
\VerbatimFootnotes

\chapter{Related work} \label{ch:relatedwork}

\section{Other projects using Elpi}
\begin{itemize}
    \item \cite{tassiDerivingProvedEquality2019} Derive
    \item \cite{blotCompositionalPreprocessingAutomated2023} Trakt
    \item \cite{cohenHierarchyBuilderAlgebraic2020} Hierarchy builder
\end{itemize}

\section{Fixpoints}
\begin{itemize}
    \item \cite{brotherstonFormalisedInductiveReasoning2007,brotherstonCyclicProofsProgram2008} Defines inductive predicates in a BI logic \quest{What is this exactly? I know Iris is one?}. Follows simliar steps, fiven a pre fixpoint function, called an operator. Proves it monotone and then takes the least fixpoint. \quest{I don't understand how it generates the function, because it seems to do so.}
\end{itemize}

\section{Inductive predicates in other separation logic proof systems}

\begin{itemize}
    \item \cite{caoVSTFloydSeparationLogic2018} mentions representation predicates and shows how to define simple variants. Only works when it has a strictly decreasing recursive call.
    \item \cite{berdineSmallfootModularAutomatic2005} is a tool for checking a subset of separation logic specifications. It includes several built in representation predicates. You are not able to define more.
    \item \cite{leinoDafnyAutomaticProgram2010} uses an SMT based automatic program verifier, with added hints if necessary. It automatically can handle recursive data types and do induction on them.
    \item \cite{filliatreOneLogicUse2013} They support inductive predicates. But doesn't use separation logic I think.
    \item \cite{mullerViperVerificationInfrastructure2016,summersAutomatingDeductiveVerification2018} Viper, is based on separation logic. It allows for automatic definition of inductive predicates. Not clear how, never explained as far as I can tell.
    \item \cite{jacobsVeriFastPowerfulSound2011} is based on separation logic.
\end{itemize}

\section{Generalized rewriting, Propers}
\begin{itemize}
    \item \cite{sozeauNewLookGeneralized2009} Generalized rewriting. They backtrack in their proof search, we do not.
\end{itemize}


\end{document}