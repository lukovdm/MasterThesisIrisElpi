\documentclass[aspectratio=169]{beamer}

\usepackage[no-math]{fontspec}
% \usepackage{microtype}
\setmonofont{Hack}
% \newfontfamily{\fallbackfont}{DejaVu Sans Mono}
% \DeclareTextFontCommand{\textfallback}{\fallbackfont}
% \newcommand{\fallbackchar}[2][\textfallback]{%
%     \newunicodechar{#2}{#1{#2}}%
% }
% \fallbackchar{⌜}
% \fallbackchar{⌝}
 
% \usepackage{silence} 
% \WarningsOff[latexfont]

\usepackage[rgb]{xcolor}
\usepackage{url}

\usepackage{graphicx}
\usepackage[style=alphabetic, maxbibnames=99]{biblatex}
\usepackage{refcount}
\usepackage{array}
\usepackage{mathpartir}
\usepackage{pftools}
\usepackage{iris}
\usepackage{heaplang}
\usepackage{thesiscommands}

\usetheme{Frankfurt}
\usecolortheme{beaver}

\title{Extending the Iris Proof Mode with Inductive Predicates using Elpi}
\author{Luko van der Maas}
\institute{Computing Science\\Radboud University}
\date{}
 

\begin{document}
\frame{\titlepage}

\begin{frame}[fragile]{The goal}
    \begin{center}
        \begin{tikzpicture}
            \node [MLL] (x0) at (0,0) {$\val_0$};
            \node [MLL, marked] (x1) at (3,0) {$\val_1$};
            \node [MLL] (x2) at (6,0) {$\val_2$};
            \node [MLL, null] (x3) at (9,0) {$\val_3$};

            \path[{Circle}->,thick] ([yshift=1pt,xshift=1pt]x0.three) edge [bend left] (x1.west);
            \path[{Circle}->,thick] ([yshift=1pt,xshift=1pt]x1.three) edge [bend left] (x2.west);
            \path[{Circle}->,thick] ([yshift=1pt,xshift=1pt]x2.three) edge [bend left] (x3.west);

            \node (l) [above=of x0.west] {$\loc$};
            \path[->,thick] (l) edge ([yshift=.1cm]x0.north west);
        \end{tikzpicture}
    \end{center}

    \begin{coqcode}
        eiInd
        Inductive is_MLL : val → list val → iProp :=
            | empty_is_MLL : is_MLL NONEV []
            | mark_is_MLL v vs l tl : 
              l ↦ (v, #true, tl) -∗ is_MLL tl vs -∗ 
              is_MLL (SOMEV #l) vs
            | cons_is_MLL v vs tl l : 
              l ↦ (v, #false, tl) -∗ is_MLL tl vs -∗ 
              is_MLL (SOMEV #l) (v :: vs).
      \end{coqcode}

\end{frame}

\section{Recap on program verification}

\begin{frame}{Program verification}
    \begin{itemize}
        \item Verify programs by specifying pre and post conditions
        \item Specification happens in separation logic
    \end{itemize}
\end{frame}

\begin{frame}{Separation logic}
    \begin{center}
        \begin{tikzpicture}
            \node[memnode] (x) at (0,0) {$\val$};
            \node (l) [above=of x.west] {$\loc$};

            \path[->, thick] (l) edge ([yshift=.1cm]x.north west);
            \path (0.5,-.8) edge[dashed] node[fill=white] {\large$\ast$} (0.5,1.8);
            \begin{scope}[xshift=1.2cm]
                \node[memnode] (y) at (0,0) {$\valB$};
                \node (k) [above=of y.west] {$\locB$};

                \path[->, thick] (k) edge ([yshift=.1cm]y.north west);
            \end{scope}
            \node (logic) at (4,0.8) {$\loc\fmapsto\val\ \ast\ \locB\fmapsto\valB$};
        \end{tikzpicture}
    \end{center}
    \[\hoare{\isMLL\, \hd\, \vect{\val}}{\MLLdelete\, \hd\; \iindex }{\isMLL\, \hd\; (\textlog{remove}\, \iindex\, \vect{\val})}\]
\end{frame}

\begin{frame}{Representation predicates}
    \begin{align*}
        \isMLL\, \hd\, \vect{\val} & =
        \isMLLRecDef
    \end{align*}
    \begin{mathpar}
        \inferhref{\isMLL-ind}{ismll-ind}
        {\TRUE\proves \pred\, \None\, []
            \\
            \loc \fmapsto (\val', \True, \tl) * (\isMLL\, \tl\, \vect{\val} \land \pred\, \tl\, \vect{\val}) \proves \pred\, (\Some \loc)\, \vect{\val}
            \\
            \loc \fmapsto (\val', \False, \tl) * (\isMLL\, \tl\, \vect{\val} \land \pred\, \tl\, \vect{\val}) \proves \pred\, (\Some \loc)\, (\val' :: \vect{\val})
        }
        {\isMLL\, \hd\, \vect{\val} \proves \pred\, \hd\, \vect{\val}}
    \end{mathpar}
\end{frame}

\section{Problem statement}
\begin{frame}{Problem}
    \begin{itemize}
        \item $\isMLL$ can't just be defined because Coq
        \item \ldots
    \end{itemize}
\end{frame}

\section{Outline of system}
\begin{frame}[fragile]{Inductive command}
    \begin{coqcode}
        eiInd
        Inductive is_MLL : val → list val → iProp :=
            | empty_is_MLL : is_MLL NONEV []
            | mark_is_MLL v vs l tl : 
              l ↦ (v, #true, tl) -∗ is_MLL tl vs -∗ 
              is_MLL (SOMEV #l) vs
            | cons_is_MLL v vs tl l : 
              l ↦ (v, #false, tl) -∗ is_MLL tl vs -∗ 
              is_MLL (SOMEV #l) (v :: vs).
    \end{coqcode}
\end{frame}

\begin{frame}[fragile]{Tactics}
    \begin{itemize}
        \item \coqi{eiInduction}
        \item \coqi{eiIntros} \& \coqi{eiDestruct}
        \item Lemmas: \coqi{empty_is_MLL}, \coqi{mark_is_MLL}, \coqi{cons_is_MLL}
    \end{itemize}
\end{frame}

\begin{frame}{Demo}
    Demo of delete on $is_MLL$
\end{frame}

\section{Fixpoints}
\begin{frame}{Outline of construction}
    \begin{itemize}
        \item Transform inductive definition into a pre fixpoint function
        \item Prove that pre fixpoint function is monontone
        \item Take least fixpoint of pre fixpoint function
    \end{itemize}
\end{frame}

\begin{frame}{Pre fixpoint function}

\end{frame}

\begin{frame}{Least fixpoint}

\end{frame}

\section[Implementation of eiInd]{Implementation of \coqi{eiInd}}
\begin{frame}{Elpi}
    \begin{itemize}
        \item $\lambda$Prolog dialect
        \item example
        \item \ldots
    \end{itemize}
\end{frame}

\begin{frame}{Coq-Elpi}

\end{frame}

\begin{frame}{Implementing tactics}

\end{frame}

\begin{frame}{Structure of \texttt{eiInd}}

\end{frame}

\section{Conclusion}
\begin{frame}{Contributions}
    \begin{itemize}
        \item Created system for defining and using inductive predicates in the IPM
        \item Strategy for modular tactics in Elpi
        \item Generating monotonicity proof of pre fixpoint function
        \item Evaluation of Elpi as meta-programming language for the IPM
    \end{itemize}
\end{frame}

\end{document}