\documentclass[thesis.tex]{subfiles}

\ifSubfilesClassLoaded{
  \externaldocument{thesis}
  \setcounter{chapter}{0}
}{}

\begin{document}

\chapter{Introduction}
\label{ch:introduction}
% Korte beschrijving van state of the art, er is seperatie logica met ...
% Het probleem met voorbeeld
% Oplossing uitleggen
% Lijstje van je contributies, ik heb x y en z gedaan en verwijzen naar hoofdstuk
% - ze moeten nieuw zijn
% - Meetbaar zijn
% - Doelvol zijn



Iris is a separation logic \cite*{jungIrisMonoidsInvariants2015a,jungHigherorderGhostState2016,krebbersEssenceHigherOrderConcurrent2017,jungIrisGroundModular2018}. It is implemented in Coq in what is called the Iris Proof Mode (IPM) \cite*{krebbersInteractiveProofsHigherorder2017,krebbersMoSeLGeneralExtensible2018}.

We implement our tactic in the $\lambda$Prolog language Elpi \cite{dunchevELPIFastEmbeddable2015,guidiImplementingTypeTheory2019}. Elpi implements $\lambda$prolog \cite{millerHigherorderLogicProgramming1986,millerUniformProofsFoundation1991,belleanneePragmaticReconstructionLProlog1999,millerProgrammingHigherOrderLogic2012} with a few additions. These additions are crucial for the workings of \ce but won't be discussed here as they are not directly used the created tactics and commands.

To use Elpi as a Coq meta programming language, there exists the Elpi Coq connector, \ce \cite{tassiElpiExtensionLanguage2018}. We use \ce to implement the Elpi variant of \coqi{iIntros}, named \coqi{eiIntros}.

\end{document}