\documentclass[thesis.tex]{subfiles}

\begin{document}
\chapter{Inductive}
\begin{itemize}
  \item What are we going to do
  \item First create functor
  \item Proof functor is Monotone
  \item Apply fixpoint to functor
  \item Proof unfold lemma's for fixpoint
  \item Proof iter lemma for fixpoint
  \item Proof induction lemma for fixpoint
\end{itemize}

\section{Functor}
\subsection{Theory}
\begin{itemize}
  \item Take a function and apply it under one level of the inductive statement
  \item Maybe draw the diagram
\end{itemize}

\subsection{Elpi}
\begin{itemize}
  \item We can also make commands
  \item What do we get as input for our commands
  \item What do we need to turn it in to
  \item Show example for is_list
\end{itemize}

\section{Monotone}
\subsection{Theory}
\begin{itemize}
  \item What are we proving, what is Monotone
  \item This can be seen as a proper
  \item We have to transform proper to Iris Propositions
  \item Respectfull, pointwise, persistent
  \item Define Proper instances for connectives
  \item How to find proper instance
  \item IProperTop
  \item Example of Proper proof
\end{itemize}
\subsection{Elpi}
\paragraph{Proper}
\begin{itemize}
  \item Write tactic for solving IProper proofs
  \item We write small tactics for different possibel steps
  \item Simple steps, for respectfull, pointwise, persistent
  \item Finishing steps for assumption and reflexive implication
  \item Apply other proper instance
  \item Find how many arguments to add to connective
  \item Lemma to get Iproper instance from IProperTop instance
  \item Apply Lemma IProper
  \item Compose till all goals proven
\end{itemize}

\paragraph{Induction for proper}
\begin{itemize}
  \item Create Proper Type for Fixpoint
  \item Add pointwise for every constructor using fold-map
  \item Add this to left and right of Respectfull with a persistent around lefthand side
  \item Apply proper solver
\end{itemize}

\section{Least fixpoint}
\subsection{Theory}
\begin{itemize}
  \item Intuition about what a least fixpoint is
  \item The least fixpoint of a functor holds for a value if for all fixpoints of the functor the value holds
  \item What does our fixpoint function create
  \item Example for is_list
\end{itemize}
\subsection{Elpi}
\begin{itemize}
  \item The basic structure is this \dots
  \item We recurse over the type of the fixpoint to introduce lambda's and foralls
  \item As the last step we add lambda's for any parameters we have
\end{itemize}

\section{Unfold lemma}
\subsection{Theory}
\begin{itemize}
  \item Allows for more easy reasoning about fixpoints by using the functor
  \item Essential in following proofs
\end{itemize}
\subsection{Elpi}

\section{Induction scheme}
\subsection{Theory}
\subsection{Elpi}

\section[iConstructor, iDestruct, iInductive]{\coqinline{iConstructor}, \coqinline{iDestruct} abd \coqinline{iInductive}}


\end{document}