\documentclass[thesis.tex]{subfiles}

\begin{document}
\chapter{Elpi implementation of Inductive}\label{ch:inductiveimpl}`'
\section{function}

\begin{itemize}
  \item We can also make commands
  \item What do we get as input for our commands
  \item What do we need to turn it in to
  \item Show example for $\textlog{isMLL}$
\end{itemize}

\section{Monotone}
\subsection{Proper}
\begin{itemize}
  \item Write tactic for solving IProper proofs
  \item We write small tactics for different possible steps
  \item Simple steps, for respectful, point-wise, persistent
  \item Finishing steps for assumption and reflexive implication
  \item Apply other proper instance
  \item Find how many arguments to add to connective
  \item Lemma to get IProper instance from IProperTop instance
  \item Apply Lemma IProper
  \item Compose till all goals proven
\end{itemize}

\subsection{Induction for proper}
\begin{itemize}
  \item Create Proper Type for fix-point
  \item Add point-wise for every constructor using fold-map
  \item Add this to left and right of respectful with a persistent around left-hand side
  \item Apply proper solver
\end{itemize}

\section{Least fix-point}
\begin{itemize}
  \item The basic structure is this \dots
  \item We recurse over the type of the fix-point to introduce lambda's and existential quantification
  \item As the last step we add lambda's for any parameters we have
\end{itemize}


\end{document}