\documentclass[11pt,a4paper]{report}

\usepackage[rgb]{xcolor}
\usepackage{url}

\usepackage{fontspec}
\setmonofont{Hack}  

\usepackage{graphicx}
\usepackage{listings}
\usepackage[a4paper]{geometry}
\usepackage[style=alphabetic]{biblatex}
% use natbib

\usepackage{refcount}
\usepackage{xr-hyper}

\usepackage{subfiles}
\usepackage[hidelinks]{hyperref}

\usepackage{array}
\setlength {\marginparwidth }{2cm}
\usepackage{mathpartir}
\usepackage{pftools}
\usepackage{iris}
\usepackage{heaplang}
\usepackage{thesiscommands}

\usepackage[noabbrev]{cleveref}

\allowdisplaybreaks
     
\makeatletter %otherwise geometry resets everything
\Gm@restore@org
\makeatother

\setlength{\itemsep}{0cm}
\setlength{\voffset}{0cm}
\setlength{\headheight}{0cm}
\setlength{\topmargin}{0cm} 
\lstset{basicstyle = \footnotesize, breaklines = true}

\graphicspath{{imgs/}}

\addbibresource{iris-elpi.bib}

\setlength {\marginparwidth }{4cm}
\begin{document}
\renewcommand{\sectionautorefname}{Section}
\renewcommand{\subsectionautorefname}{Section}
\renewcommand{\chapterautorefname}{Chapter}

\begin{titlepage}
  \begin{center}
    \textsc{\LARGE Master thesis\\Computing Science}\\[1.5cm]
    \includegraphics[height=100pt]{logo}

    \vspace{0.4cm}
    \textsc{\Large Radboud University}\\[1cm]
    \hrule
    \vspace{0.4cm}
    \textbf{\huge Extending Iris with Inductive predicates using Elpi}\\[0.4cm]
    \hrule
    \vspace{2cm}
    \begin{minipage}[t]{0.45\textwidth}
      \begin{flushleft} \large
        \textit{Author:}\\
        Luko van der Maas\\
        \texttt{luko.vandermaas@ru.nl}\\
        s1010320
      \end{flushleft}
    \end{minipage}
    \begin{minipage}[t]{0.45\textwidth}
      \begin{flushright} \large
        \textit{Supervisor:}\\
        dr. Robbert Krebbers\\
        \texttt{robbert@cs.ru.nl}\\[1.3cm]
        \textit{Assessor:}\\
        \dots\\
        \texttt{\dots}
      \end{flushright}
    \end{minipage}
    \vfill
    {\large \today}
  \end{center}
\end{titlepage}

\begin{abstract}
    %Field, current gap, direction of solution, Results, Genererailzation of results and where else to apply it.
    Separation logic is a framework for defining and proving specifications of imperative and concurrent programs. In separation logic, representation predicates are used to relate a data structure in the heap to an object in the logic. When dealing with recursive data structures such as linked lists or trees, inductive representation predicates are needed to represent them.

    One approach to software verification systems is embedding separation logic into the logic of a proof assistant. Consequently, inductive predicates must be derived from the proof assistant logic. Three derivations exist, each using the fixpoint of a function, each having downsides. Both using the structural recursion of the proof assistant logic and using Banach fixpoints impose significant limitations on the inductive predicates that can be defined. Using the least fixpoint, on the other hand, imposes many manual proofs.

    This thesis develops a command and tactics to automate inductive predicates using the least fixpoint in the Iris framework for separation logic \cite{jungIrisGroundModular2018}, which is embedded in the Coq proof assistant as the Iris Proof Mode (IPM) \cite{krebbersMoSeLGeneralExtensible2018}. We use the Coq meta-programming language \ce \cite{tassiElpiExtensionLanguage2018} to generate the least fixpoint and prove the induction principle of this inductive predicate. Furthermore, we introduce tactics that automate applying the inductive predicate. Lastly, we use our system of commands and tactics to redefine a complicated real world inductive predicate, the total weakest precondition from Iris.

    During the creation of our system, we reimplement a significant part of the tactics from the IPM and evaluate if Elpi would be a good fit for reimplementing the full IPM.
\end{abstract}

\tableofcontents

\chapter{Introduction}
\label{ch:introduction}
Korte beschrijving van state of the art, er is seperatie logica met ...
Het probleem met voorbeeld
Oplossing uitleggen
Lijstje van je contributies, ik heb x y en z gedaan en verwijzen naar hoofdstuk
- ze moeten nieuw zijn
- Meetbaar zijn
- Doelvol zijn

\iris is a separation logic \cite*{jungIrisMonoidsInvariants2015a,jungHigherorderGhostState2016,krebbersEssenceHigherOrderConcurrent2017,jungIrisGroundModular2018}. It is implemented in \coq in what is called the Iris Proof Mode (IPM) \cite*{krebbersInteractiveProofsHigherorder2017,krebbersMoSeLGeneralExtensible2018}.

\subfile{IrisBackground}
\subfile{Fixpoints}

\subfile{ElpiTactics} %Iris Proofmode in Elpi

% \subfile{Induction}
\subfile{ElpiImplementation}

\chapter{Conclusion}
\section{Application}
\section{Evaluation of Elpi}
\section{Related work}
\section{Future work}


\printbibliography

\newpage
\mbox{}
\thispagestyle{empty}
\newpage
\mbox{}
\thispagestyle{empty}
\end{document}